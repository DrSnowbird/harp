\begin{document}

%\title{MapReduce Based Subgraph Counting: variations and performance study}
\title{Finding and counting tree-like subgraphs using MapReduce}

\author{{Zhao~Zhao, Meng~Li, Guanying~Wang, Ali~Butt, Maleq~Khan, \\
	Madhav~Marathe, Judy~Qiu, Anil~Vullikanti}% <-this % stops a space

\IEEEcompsocitemizethanks{

	\IEEEcompsocthanksitem Zhao Zhao, Ali~Butt, Madhav Marathe and Anil Vullikanti are
	with the Network Dynamics and Simulation Science Laboratory, Biocomplexity Institute \&
        Department of Computer Science, Virginia Tech, VA, 24061.\protect\\
% note need leading \protect in front of \\ to get a newline within \thanks as
% \\ is fragile and will error, could use \hfil\break instead.
	E-mail: zhaozhao@vt.edu, butta@cs.vt.edu, mmarathe@vt.edu, vsakumar@vt.edu 

	\IEEEcompsocthanksitem Maleq Khan is with the Department of Electrical Engineering and Computer Science,
	Texas A\&M University-Kingsville. \protect\\ 
	E-mail: maleq.khan@tamuk.edu

	\IEEEcompsocthanksitem Meng Li is with the Computer Science Department, Indiana University. \protect\\
	Email: li526@umail.iu.edu
		
	\IEEEcompsocthanksitem Judy Qiu is with the Intelligent Systems Engineering Department, Indiana University. \protect\\
	Email: xqiu@indiana.edu

	\IEEEcompsocthanksitem Guanying Wang is working with Google Inc. \protect\\
	Email:wang.guanying@gmail.com}% <-this % stops an unwanted space

}

% note the % following the last \IEEEmembership and also \thanks - 
% these prevent an unwanted space from occurring between the last author name
% and the end of the author line. i.e., if you had this:
% 
% \author{....lastname \thanks{...} \thanks{...} }
%                     ^------------^------------^----Do not want these spaces!
%
% a space would be appended to the last name and could cause every name on that
% line to be shifted left slightly. This is one of those "LaTeX things". For
% instance, "\textbf{A} \textbf{B}" will typeset as "A B" not "AB". To get
% "AB" then you have to do: "\textbf{A}\textbf{B}"
% \thanks is no different in this regard, so shield the last } of each \thanks
% that ends a line with a % and do not let a space in before the next \thanks.
% Spaces after \IEEEmembership other than the last one are OK (and needed) as
% you are supposed to have spaces between the names. For what it is worth,
% this is a minor point as most people would not even notice if the said evil
% space somehow managed to creep in.



% The paper headers
\markboth{IEEE Transactions on Multi-Scale Computing Systems}%
{Shell \MakeLowercase{\textit{et al.}}:MapReduce Based Subgraph Counting:
Variations and Performance Study}

% The only time the second header will appear is for the odd numbered pages
% after the title page when using the twoside option.
% 
% *** Note that you probably will NOT want to include the author's ***
% *** name in the headers of peer review papers.                   ***
% You can use \ifCLASSOPTIONpeerreview for conditional compilation here if
% you desire.



% The publisher's ID mark at the bottom of the page is less important with
% Computer Society journal papers as those publications place the marks
% outside of the main text columns and, therefore, unlike regular IEEE
% journals, the available text space is not reduced by their presence.
% If you want to put a publisher's ID mark on the page you can do it like
% this:
%\IEEEpubid{0000--0000/00\$00.00~\copyright~2015 IEEE}
% or like this to get the Computer Society new two part style.
%\IEEEpubid{\makebox[\columnwidth]{\hfill 0000--0000/00/\$00.00~\copyright~2015 IEEE}%
%\hspace{\columnsep}\makebox[\columnwidth]{Published by the IEEE Computer Society\hfill}}
% Remember, if you use this you must call \IEEEpubidadjcol in the second
% column for its text to clear the IEEEpubid mark (Computer Society jorunal
% papers don't need this extra clearance.)



% use for special paper notices
%\IEEEspecialpapernotice{(Invited Paper)}



% for Computer Society papers, we must declare the abstract and index terms
% PRIOR to the title within the \IEEEtitleabstractindextext IEEEtran
% command as these need to go into the title area created by \maketitle.
% As a general rule, do not put math, special symbols or citations
% in the abstract or keywords.
\IEEEtitleabstractindextext{%

\begin{abstract}

Several variants of the subgraph isomorphism problem, e.g., finding, counting
and estimating frequencies of subgraphs in networks arise in a number of real
world applications, such as genetic network analysis in bioinformatics, web
analysis, disease diffusion prediction and social network analysis. These
problems are computationally challenging to scale to very large networks with
millions of nodes. In this paper, we present \sahad{}, a MapReduce based
algorithm for detecting and counting trees of bounded size using the elegant
color coding technique, developed by N. Alon, R. Yuster and U. Zwick, Journal of
the ACM (JACM) 1995.  \sahad{} is a randomized algorithm, and we show rigorous
bounds on the approximation quality and the performance. We implement \sahad{}
on two different frameworks: the standard Hadoop model and Harp, a more high
performance computing environment, and evaluate its performance on a variety of
synthetic and real networks. \sahad{} scales to very large networks comprising
of $10^7-10^8$ nodes and $10^8-10^9$ edges and tree-like (acyclic) templates
with up to 12 nodes.  Further, we extend our results by implementing our
algorithm using the Harp framework.  The new implementation gives an order of
magnitude improvement in performance over the standard Hadoop implementation.


\end{abstract}

% Note that keywords are not normally used for peerreview papers.
\begin{IEEEkeywords}
subgraph isomorphism, graph partitioning, MapReduce, Hadoop, Harp
\end{IEEEkeywords}}


% make the title area
\maketitle


% To allow for easy dual compilation without having to reenter the
% abstract/keywords data, the \IEEEtitleabstractindextext text will
% not be used in maketitle, but will appear (i.e., to be "transported")
% here as \IEEEdisplaynontitleabstractindextext when the compsoc 
% or transmag modes are not selected <OR> if conference mode is selected 
% - because all conference papers position the abstract like regular
% papers do.
\IEEEdisplaynontitleabstractindextext
% \IEEEdisplaynontitleabstractindextext has no effect when using
% compsoc or transmag under a non-conference mode.



% For peer review papers, you can put extra information on the cover
% page as needed:
% \ifCLASSOPTIONpeerreview
% \begin{center} \bfseries EDICS Category: 3-BBND \end{center}
% \fi
%
% For peerreview papers, this IEEEtran command inserts a page break and
% creates the second title. It will be ignored for other modes.
\IEEEpeerreviewmaketitle

